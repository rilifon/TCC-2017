% ---------------------------------------------------------------------------- %
\chapter{Introdução}
\label{cap:introducao}
% ---------------------------------------------------------------------------- %

    Desenvolvimento de jogos é uma área de estudos interdisciplinar onde podemos aplicar conhecimentos de arte, música, \textit{Game Design} e, especialmente em jogos digitais, ciência da computação. Não é uma área que falta desafios e problemas a serem solucionados, e a produção de um jogo requer a aplicação direta de muitos conceitos e práticas ensinadas durante o curso em Ciência da Computação do Instituto de Matemática e Estatística da Universidade de São Paulo.

    Em novembro de 2009 foi fundado o \textbf{UspGameDev}, o grupo extracurricular da Universidade de Sâo Paulo dedicado a fazer a ponte entre interessados em jogos e estudantes da faculdade, em especial estudantes da computação. Aberto a toda comunidade Uspiana, a \textit{UspGameDev} ou \textit{UGD} sempre ficou de portas abertas para qualquer aluno que gostaria de aprender mais sobre desenvolvimento de jogos, e ter a chance de se reunir com outras pessoas e produzir seus próprios jogos, sejam eles digitais ou analógicos.

    Em 2013, no meu primeiro ano de faculdade entrei na \textit{UGD} e junto de um colega fiz meu primeiro jogo virtual: \textit{PsyChObALL}, um \textit{top-down shooter psicodélico}. Foi o primeiro "grande" jogo que produzi durante a faculdade, e isso me inspirou a continuar estudando a área de desenvolvimento de jogos, sempre conectado à \textit{UspGameDev}.

    Foi com essa mentalidade que decidi, no meio da graduação, produzir um \textit{remake} de \textit{PsyChObALL} chamado de \textit{PsyChO: The Ball}, juntando todo o conhecimento adquirido durante os anos de estudos em computação e design de jogos.


% ---------------------------------------------------------------------------- %
\section{Motivação}
\label{sec:motivacao}

    


% ---------------------------------------------------------------------------- %
\section{Objetivos}
\label{sec:objetivo}

    Texto texto texto texto texto texto texto texto texto texto texto texto
    texto texto texto texto texto texto texto texto texto texto texto texto
    texto texto texto texto texto texto texto.

    Texto texto texto texto texto texto texto texto texto texto texto texto
    texto texto texto texto texto texto texto texto texto texto texto texto.
