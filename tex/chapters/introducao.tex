% ---------------------------------------------------------------------------- %
\chapter{Introdução}
\label{cap:introducao}
% ---------------------------------------------------------------------------- %

  Desenvolvimento de jogos é uma área de estudos interdisciplinar onde podemos aplicar conhecimentos de arte, música, \textit{Game Design} e, especialmente em jogos digitais, ciência da computação. Não é uma área que falta desafios e problemas a serem solucionados, e a produção de um jogo requer a aplicação direta de muitos conceitos e práticas ensinadas durante o curso em Ciência da Computação do Instituto de Matemática e Estatística da Universidade de São Paulo.

  Em novembro de 2009 foi fundado o \textbf{UspGameDev}, o grupo extracurricular da Universidade de Sâo Paulo dedicado a fazer a ponte entre interessados em jogos e estudantes da faculdade, em especial estudantes da computação. Aberto a toda comunidade Uspiana, a \textit{UspGameDev} ou \textit{UGD} sempre ficou de portas abertas para qualquer aluno que gostaria de aprender mais sobre desenvolvimento de jogos, e ter a chance de se reunir com outras pessoas e produzir seus próprios jogos, sejam eles digitais ou analógicos.

  Em 2013, no meu primeiro ano de faculdade entrei na \textit{UGD} e junto de um colega fiz meu primeiro jogo virtual: \textit{PsyChObALL}, um \textit{top-down shooter psicodélico}. Foi o primeiro "grande" jogo que produzi durante a faculdade, e isso me inspirou a continuar estudando a área de desenvolvimento de jogos, sempre conectado à \textit{UspGameDev}.

  Foi com essa mentalidade que decidi, no meio da graduação, com início na matéria \textit{Atividade Curricular em Cultura e Extensão}, produzir um \textit{remake} (termo análogo a uma "refilmagem" aplicado à jogos) de \textit{PsyChObALL} chamado de \textit{PsyChO: The Ball}, juntando todo o conhecimento adquirido durante os anos de estudos em computação e design de jogos.


% ---------------------------------------------------------------------------- %
\section{Motivação e Objetivos}
\label{sec:motivacao_objetivo}

  O objetivo central desse trabalho final de formatura é aplicar todo o conhecimento que aprendi durante a faculdade focado na área de desenvolvimento de jogos, ou \textit{"game dev"}, e utilizar esse projeto como mais uma fonte de estudos e aprendizados. Logo foi de interesse atravessar durante todas etapas de desenvolvimento de um jogo digital.

  Primeiramente é preciso a discussão de \textit{balanço} sobre o jogo original \textit{PsyChObALL}, analisando seus pontos positivos e negativos. A segunda etapa seria construir uma base sólida para rodar o jogo, utilizando meus conhecimentos para criar bibliotecas e um ambiente apropriado para a construção do jogo. A terceira etapa seria o desenvolvimento do jogo em sí, utilizando como referência o jogo original. Por último seria a etapa de testar com usuários e receber \textit{feedback}, podendo assim voltar à etapa 3 de desenvolver o jogo melhorando o que for necessário e repetir o ciclo.

  É esperado que depois de várias iterações cheguemos num protótipo jogável e divertido de se jogar de \textit{PsyChO: The Ball}, para que futuramente possa disponibilizar o jogo em alguma plataforma de distribuição de jogos digitais.
