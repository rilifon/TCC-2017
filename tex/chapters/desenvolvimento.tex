% ---------------------------------------------------------------------------- %
\chapter{Desenvolvimento}
\label{cap:desenvolvimento}
% ---------------------------------------------------------------------------- %

\textit{PsyChO: The Ball} teve seu início em 2013 como \textit{PsyChObALL}. Durante os primeiros meses o projeto servia como um meio prático de se aprender técnicas de programação e desenvolvimento de jogos, e aos poucos foi tomando proporções maiores até ter seu primeiro lançamento em agosto de 2013. O jogo ficou em desenvolvimento até meio de 2014, mas lentamente começou a receber menos atualizações conforme seus criadores foram focando em outros projetos.

Somente no segundo semestre de 2015 que ele ressurgiu como \textit{PsyChO: The Ball} na disciplina \textbf{MAC0214 - Atividade Curricular em Cultura e Extensão}. O objetivo do projeto era revitalizar o \textit{PsyChObALL} original, fazendo tudo do zero e utilizando todo conhecimento acumulado durante a graduação. O fim de 2016 teve como resultado um protótipo bem elaborado do jogo com todas mecânicas originais do \textit{PsyChObALL} implementadas e melhoradas.

O desenvolvimento do jogo passou por várias fases. Segue neste capítulo alguma das infraestruturas mais importantes desenvolvidas durante e para o projeto.

% ---------------------------------------------------------------------------- %
\section{STEAMING}
\label{sec:steaming}

Antes de começar o desenvolvimento direto do jogo, os primeiros meses foram direcionados em criar um template de jogos para o arcabouço \textit{Love2D}. Esse template se tornou o \textit{STEAMING} (\textit{Simple TEmplAte for MakINg Games}), e todo o código do jogo foi construído em cima dele.

O template possue duas características peculiares que foram vantajosas para o desenvolvimento do jogo: sua infraestrutura orientada objetos para manipular elementos num jogo e sua estrutura de desenhar objetos na tela.

Todos objetos no jogo herdam de uma classe básica chamada \textit{Element}. Um \textit{Element} pode possuir três atributos fundamentais: um tipo (chamado de \textit{type}), um subtipo (chamado de \textit{subtype}) e uma identificação única (chamada de \text{id}). Com esses três atributos é possível agrupar, identificar, selecionar ou até mesmo destruir elementos no jogo com muita facilidade, facilitando imensamente o desenvolvimento de jogos.

\textit{STEAMING} reduz todo pipeline gráfico do jogo em camadas. Elementos do jogo desenháveis são atribuidos a alguma camada, e a própria infraestrutura do template vai percorrer todas elas  em ordem e desenhar os objetos dentro de cada uma, chamando o método correspondente da classe. Com essa organização é possível diagramar o design gráfico do jogo bem mais facilmente, e não se preocupar com objetos sendo desenhados na ordem certa. Entretanto objetos na mesma camada são desenhados em uma ordem imprevisível, já que são armazenados dentro de uma tabela \textit{hash} sem ordem específica.

% ---------------------------------------------------------------------------- %
\section{Gerenciador de Saves}
\label{sec:gerenciador_de_saves}

Para armazenar dados entre partidas, foi desenvolvido um gerenciador de saves. Este é responsável por armazenar as informações relevantes do jogo, como maiores pontuações, configurações de som escolhidas pelo usuário, se é a primeira vez que o jogador abriu o jogo, etc.

Todas informações são guardadas em tabelas de lua, que depois são transformadas no formato JSON e salvas no disco rígido. Para recuperar essas informações, o gerenciador de saves faz a transformação inversa toda vez que o jogo é inicializado e atribui os valores para suas respectivas variáveis.

% ---------------------------------------------------------------------------- %
\section{Script de Fases}
\label{sec:level_script}

Para organizar o fluxo de eventos em cada nível do jogo, foi implementado um gerenciador de niveis. Assim foi possível abstrair a lógica de cada fase em scripts.

Cada script descreve em que sequência inimigos ou ações se desenrolam em cada nível do jogo. Isso facilitou imensamente no design dos níveis, especialmente quando foi necessário balancear o jogo, pois o script permite editar e visualizar facilmente cada nível em poucas linhas. Vários métodos foram criados para gerar padrões específicos de inimigos, dar vidas-extras para o Psycho ou até mesmo criar chefões.

Porém, para manter um ritmo durante o nível, e não fazer o gerenciador rodar todas as linhas de uma vez, foi necessário usar corotinas. Elas permitem paralelizar o código e gerar múltiplos pontos de retorno para que o jogo aguarde certas condições antes de continuar o script de um nível. Essas condições pode ser de tempo (como esperar 10 segundos antes de enviar a próxima sequência de inimigos) ou de aguardar até que não tenha um inimigo vivo na tela (utilizada por exemplo para esperar que o jogador destrua todos inimigos antes de avançar para o nível seguinte).

Para lidar com corotinas paradas ou criar novas corotinas, vários métodos auxiliares foram implementadas, sempre com o objetivo de abstrair o máximo possível para que mesmo um não-programador consiga criar seus próprios níveis no jogo.

% ---------------------------------------------------------------------------- %
\section{Shaders}
\label{sec:shaders}

Uma boa descrição de \textit{PsyChO: The Ball} seria: contém muitos círculos. Logo, como este é o elemento mais característico do jogo, é desejável que os círculos sejam visualmente agradáveis.

Inicialmente os círculos eram desenhados com os métodos imbutidos no arcabouço \textit{LÖVE}. Entretanto estes não utilizam nenhum algoritmo de suavemento, então as bordas dos circulos ficam serrilhadas e não combinam com a estética do jogo.

Desta forma escrevemos um \textit{shader} que desenha círculos com bordas suavizadas (ou como é conhecido esse efeito, \textit{anti-aliasing}). Utilizando a linguagem de \textit{shaders} da \textit{LÖVE} (muito semelhante à linguagem de \textit{shaders} do \textit{OpenGL}, \textit{GLSL}), um script genêrico consegue desenhar qualquer circulo com bordas suavizadas dado o tamanho do seu diâmetro.

Além disso o jogo precisa de outro efeito realizado por \textit{shaders}: borrar a tela, ou \textit{blur}. Esse efeito acontece quando o jogador pausa o jogo, de forma que somente a interface da tela de pause fique nítida, enquanto todos outros elementos do jogo fiquem desfocados como se estivessem borrados.

Para atingir tal efeito criamos dois scripts de \textit{shader}, um que aplica um efeito de borrão horizontal, e outro vertical. Assim ao utilizar os dois em sequência é possível chegar no resultado esperado de uma tela desfocada, criando um efeito muito legal enquanto o jogo estiver pausado.

% ---------------------------------------------------------------------------- %
\section{Juiceness}
\label{sec:juiceness}

Além desses efeitos com shaders, um grande foco no desenvolvimento do \textit{PsyChO: The Ball} foi gasto em deixar o jogo mais \textit{juicy}. Isso é realizado com vários efeitos pequenos mas que tentam melhorar a experiência do jogador.

Quase todas transições do jogo são feitas com alguma interpolação em vez de instantâneas. Um bom exemplo disso é a movimentação do personagem principal, \textit{Psycho}. Em vez de ele ter uma velocidade constante quando se move, o Psycho tem uma leve aceleração quando começa seu movimento, e um leve desaceleração quando o jogador solta o comando de andar. Isso cria uma ilusão de movimento muito mais realística e agradável pro jogador em contraposto a uma velocidade que se inicia e acaba instantaneamente.

Outro efeito implementado no jogo para aumentar a \textit{juiceness} é a explosão de partículas toda vez que um inimigo morre. Vários pequenos circulos de tamanho e velocidade variada explodem assim que um inimigo é destruído, dando mais emoção ao jogo e satisfação ao jogador.

Como último exemplo de efeito \textit{juicy} no jogo, \textit{PsyChO: The Ball} utiliza um clássico em jogos hoje em dia: tremer a tela quando o jogador morre. Outro efeito bem simples de se implementar, já que só é preciso transladar todos elementos na tela alguns pixels para a frente e para trás, porém que gera grande impacto e imersão ao jogador.

% ---------------------------------------------------------------------------- %
\section{Problemas e Desafios}
\label{sec:problemas_e_desafios}

Durante o desenvolvimento de \textit{PsyChO: The Ball} surgiram vários problemas e desafios. Para consertar bugs ou otimizar o código foi necessário aprender a fundo a linguagem lua, explorar bibliotecas disponíveis e até mesmo relembrar algoritmos aprendidos na graduação.

Um dos desafios mais interessantes que surgiu durante o desenvolvimento do jogo foi a transição de cor dos objetos. Para manter o tema psicodélico, quase todos objetos ficam transitando entre uma tabela de cores própria, porém essa transição nunca parecia natural e orgânica já que eu apenas interpolava os valores \textit{RGB} de um objeto para os valores alvo. Para resolver isso aprendi a forma de representar cores \textit{HSL}, que representa uma cor através de sua matiz, saturação e luminosidade. Com essa mudança, o jogo ficou com transições entre cores muito mais naturais e visualmente belas.
