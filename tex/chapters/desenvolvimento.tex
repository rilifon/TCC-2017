% ---------------------------------------------------------------------------- %
\chapter{Desenvolvimento}
\label{cap:desenvolvimento}
% ---------------------------------------------------------------------------- %

\textit{PsyChO: The Ball} teve seu início em 2013 como \textit{PsyChObALL}. Durante os primeiros meses, o projeto servia como um meio prático de se aprender técnicas de programação e desenvolvimento de jogos e, aos poucos, foi tomando proporções maiores até ter seu primeiro lançamento em agosto de 2013. O jogo continuou em desenvolvimento até meados de 2014, mas lentamente começou a receber menos atualizações conforme seus criadores foram focando em outros projetos.

Foi somente no segundo semestre de 2015 que ele ressurgiu como \textit{PsyChO: The Ball}, na disciplina \textbf{MAC0214 - Atividade Curricular em Cultura e Extensão}. O objetivo do projeto era revitalizar o \textit{PsyChObALL} original, fazendo tudo do início e utilizando todo conhecimento acumulado durante a graduação. O fim de 2016 teve como resultado um protótipo bem elaborado do jogo, com todas mecânicas originais do \textit{PsyChObALL} implementadas e melhoradas.

O desenvolvimento do jogo passou por várias fases. Seguem neste capítulo algumas das infraestruturas mais importantes desenvolvidas para o projeto.

% ---------------------------------------------------------------------------- %
\section{STEAMING}
\label{sec:steaming}

Antes de começar o desenvolvimento direto de \textit{PsyChO: The Ball}, os primeiros meses foram direcionados em criar um template de jogos para o arcabouço \textit{Love2D}. Esse template se tornou o \textbf{STEAMING} (\textit{Simple TEmplAte for MakINg Games}) e todo o código do jogo foi construído em cima dele.

\textbf{STEAMING} possui duas características peculiares que foram vantajosas para o desenvolvimento do jogo: sua infraestrutura orientada a objetos para manipular elementos num jogo e sua estrutura para desenhar objetos na tela.

Todos os objetos no jogo foram herdados de uma classe básica chamada \textit{Element}. Um \textit{Element} pode possuir três atributos fundamentais: um tipo (chamado de \textit{type}), um subtipo (chamado de \textit{subtype}) e uma identificação única (chamada de \textit{id}). Com esses três atributos é possível agrupar, identificar, selecionar ou até mesmo destruir elementos no jogo com muita praticidade, ajudando imensamente seu desenvolvimento.

\textbf{STEAMING} reduz todo o \textit{pipeline gráfico} do jogo em camadas. Elementos do jogo desenháveis são atribuídos à alguma camada e a própria infraestrutura do template vai percorrer todas elas, em ordem, e desenhar os objetos dentro de cada uma, chamando o método correspondente da classe. Com essa organização é possível diagramar o \textit{design gráfico} do jogo bem mais facilmente e não se preocupar com objetos sendo desenhados na ordem certa. Entretanto, objetos na mesma camada são desenhados em uma ordem imprevisível, já que são armazenados dentro de uma \textit{tabela hash} sem ordem específica.

% ---------------------------------------------------------------------------- %
\section{Gerenciador de Saves}
\label{sec:gerenciador_de_saves}

Para armazenar dados entre partidas, foi desenvolvido um \textbf{gerenciador de saves}. Este é responsável por guardar as informações relevantes do jogo, como maiores pontuações, configurações de som escolhidas pelo usuário, se é a primeira vez que o jogador abriu o jogo, entre outras.

Todas as informações são guardadas em tabelas de Lua, que depois são transformadas no formato padronizado \textit{JSON} e salvas no disco rígido. Para recuperar essas informações, o \textbf{gerenciador de saves} faz a transformação inversa toda vez que o jogo é inicializado e atribui os valores para suas respectivas variáveis.

% ---------------------------------------------------------------------------- %
\section{Script de Fases}
\label{sec:level_script}

Para organizar o fluxo de eventos em cada nível do jogo, foi implementado um gerenciador de níveis. Assim foi possível abstrair a lógica de cada fase em \textit{scripts}.

Cada \textit{script} descreve em que sequência inimigos ou ações se desenrolam em cada nível do jogo. Isso facilitou imensamente no \textit{design} dos níveis, especialmente quando foi necessário balancear o jogo, pois o \textit{script} permite editar e visualizar facilmente cada nível em poucas linhas. Vários métodos foram criados para gerar padrões específicos de inimigos, dar vidas-extras para o Psycho ou até mesmo criar chefões.

Porém, para manter um ritmo durante o nível e não fazer o gerenciador rodar todas as linhas de uma vez, foi necessário usar corotinas. Elas permitem paralelizar o código e gerar múltiplos pontos de retorno para que o jogo aguarde certas condições antes de continuar o \textit{script} de um nível. Essas condições podem ser de tempo (como esperar 10 segundos antes de enviar a próxima sequência de inimigos) ou de aguardar até que não tenha um inimigo vivo na tela (utilizada, por exemplo, para esperar que o jogador destrua todos inimigos antes de avançar para o nível seguinte).

Para lidar com corotinas paradas, ou criar novas corotinas, vários métodos auxiliares foram implementados, sempre com o objetivo de abstrair o máximo possível para que mesmo um não-programador consiga criar seus próprios níveis no jogo.

% ---------------------------------------------------------------------------- %
\section{Shaders}
\label{sec:shaders}

Uma boa descrição de \textit{PsyChO: The Ball} seria: contém muitos círculos. Logo, como este é o elemento mais característico do jogo, é desejável que os círculos sejam visualmente agradáveis.

Inicialmente os círculos eram desenhados com os métodos imbutidos no arcabouço \textit{LÖVE}. Entretanto estes não utilizam nenhum algoritmo de suavização, como consequência as bordas dos círculos ficam serrilhadas e não combinam com a estética do jogo.

Desta forma escrevemos um \textit{shader} que desenha círculos com bordas suavizadas (ou, como é conhecido esse efeito, \textit{anti-aliasing}). Utilizando a linguagem de \textit{shaders} da \textit{LÖVE} (muito semelhante à linguagem de \textit{shaders} do \textit{OpenGL}, \textit{GLSL}), um \textit{script} genérico consegue desenhar qualquer círculo com bordas suavizadas, dado o tamanho do seu diâmetro.

Além disso, o jogo precisa de outro efeito realizado por \textit{shaders}: borrar a tela, denominado \textit{blur}. Esse efeito acontece quando o jogador pausa o jogo, de forma que somente a interface da tela de pausa fique nítida, enquanto todos outros elementos do jogo fiquem desfocados como se estivessem borrados.

Para atingir tal efeito foi criado dois \textit{scripts} de \textit{shader}, um que aplica um efeito de borrão horizontal e outro vertical. Assim, ao utilizar os dois em sequência, é possível chegar ao resultado esperado de uma tela desfocada, criando um efeito muito interessante enquanto o jogo estiver pausado.

%colocar foto do blur

% ---------------------------------------------------------------------------- %
\section{Juiceness}
\label{sec:juiceness}

Além desses efeitos com \textit{shaders}, um grande foco no desenvolvimento do \textit{PsyChO: The Ball} foi em deixar o jogo mais \textit{juicy}. Isso é realizado com vários efeitos pequenos mas que tentam melhorar a experiência do jogador. \textit{Juicy} vem do termo \textit{Juiciness}, criado pelos desenvolvedores de jogos Martin Jonasson e Petri Purho, em uma palestra sobre \textit{Game Design}\cite{martinpetri}. Nela eles definem \textit{Juiciness} como uma recompensa ou fortalecimento de uma experiência, dada alguma ação do usuário e geralmente transmitida por efeitos especiais, tais como explosões, tremer a tela, ou até mesmo por efeitos sonoros.

Além disso, é considerado um efeito \textit{Juicy}, a utilização de interpolações e transições em vez de mudanças brutas de valores. Quase todas as transições do jogo são feitas desta forma. Um bom exemplo disso é a movimentação do personagem principal, \textit{Psycho}. Em vez dele ter uma velocidade constante quando se move, o Psycho tem uma leve aceleração quando começa seu movimento e uma leve desaceleração quando o jogador solta o comando de andar. Isso cria uma ilusão de movimento muito mais realista e agradável para o jogador em contraponto a uma velocidade que se inicia e acaba instantaneamente.

Outro efeito implementado no jogo, para aumentar a \textit{Juiciness}, é a explosão de partículas toda vez que um inimigo morre. Vários pequenos círculos de tamanho e velocidade variados explodem assim que um inimigo é destruído, dando mais emoção ao jogo e satisfação ao jogador.

Como último exemplo de efeito \textit{juicy}, \textit{PsyChO: The Ball} utiliza um clássico nos jogos de hoje em dia: tremer a tela quando o jogador morre. Este efeito, apesar de bem simples de se implementar (já que só é preciso transladar todos elementos na tela alguns pixels para a frente e para trás), gera um grande impacto e imersão ao jogador.

% ---------------------------------------------------------------------------- %
\section{Problemas e Desafios}
\label{sec:problemas_e_desafios}

Durante o desenvolvimento de \textit{PsyChO: The Ball} surgiram vários problemas e desafios. Para consertar \textit{bugs} ou otimizar o código foi necessário aprender a fundo a linguagem Lua, explorar bibliotecas disponíveis e até mesmo relembrar algoritmos aprendidos no curso.

Um dos desafios mais interessantes que surgiu foi a transição de cor dos objetos. Para manter o tema psicodélico, quase todos objetos ficam transitando entre valores de uma tabela de cores própria. Essa transição, porém, nunca parecia natural e orgânica já que apenas eram interpolados os valores \textit{RGB} de um objeto para os valores alvo. Para resolver este problema, foi utilizada a representação de cores \textit{HSL}, que representa uma cor através de sua matiz, saturação e luminosidade (traduzidos do inglês \textit{hue}, \textit{saturation} e \textit{lightness}). Com essa mudança, o jogo ficou com transições entre cores muito mais naturais e visualmente belas.
