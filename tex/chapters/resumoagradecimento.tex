% ---------------------------------------------------------------------------- %
\chapter*{Agradecimentos}
% ---------------------------------------------------------------------------- %

Primeiramente gostaria de agradecer aos meus pais, Valéria e Luis Edmundo, pelo apoio ininterrupto durante toda a graduação e terem me dado inúmeras oportunidades para eu ser livre em seguir o caminho que mais me desse felicidade na vida. Eu não estaria onde estou agora sem vocês.\\

Um agradecimento especial à Mayte Mirio, que aguentou todas as noites que virei programando em jogos, e nunca parou de me apoiar. Seu amor e carinho me deu forças para enfrentar inúmeros desafios na vida e me trouxe mais felicidade do que eu jamais conseguiria imaginar.\\

Quero agradecer a todos meus amigos que sempre me trouxeram alegria, não importa a situação. Felipe e Sabrina, obrigado pela disposição de baixar e testar meus joguinhos consideráveis vezes, me ajudando com críticas e me dando a satisfação de ver pessoas se divertindo com meus projetos. Rodrigo, obrigado pelas eternas noites de jogatinas, as inúmeras risadas e uma amizade eviterna. Yan e Renato, sem vocês eu não seria o aspirante desenvolvedor de jogos indies que sou hoje. Obrigado do fundo do meu coração por ter iniciado essa chama em meu coração, pelas longas Game Jams e pelas tardes de jogatinas. Espero que parte dessa chama continue em vossos corações e também nunca se apague.\\

E a todos meus outros amigos, que eu poderia listar perpetuamente, que por horas compartilharam de seu tempo comigo em atividades lúdicas, nos divertindo até altas horas, obrigado por me lembrar do verdadeiro propósito para o qual eu faço jogos.\\

A todos os professores durante a graduação que tiveram paciência e dedicação em transmitir conhecimento para mim e meus colegas, obrigado por ter aturado esse aluno desorganizado e atrapalhado por tantos anos. Espero um dia poder repassar tudo que aprendi a frente, e que minha curiosidade pelo mundo da computação e desenvolvimento de jogos que vocês iniciaram nunca se acabe.\\

Por último, nada disso seria possível sem o grupo extracurricular \textbf{UspGameDev} e todos seus membros, passados e presentes, que me acolheram desde o primeiro ano da faculdade e me ensinaram muito mais do que eu esperaria. Um grande agradecimento ao doutorando Wilson Kazuo Mizutani, fundador e membro ativo do grupo, que desde meu primeiro ano nunca ignorou qualquer dúvida ou questionamento que tive, seja sobre programação, desenvolvimento de jogos, \textit{game design} ou sobre a vida acadêmica.\\\\

A todas essas pessoas e tantas outras que marcaram minha vida, tanto dentro quanto fora da faculdade, eu dedico \textit{PsyChO: The Ball} a vocês. Obrigado por tudo.

% ---------------------------------------------------------------------------- %
\chapter*{Resumo}
% ---------------------------------------------------------------------------- %

\noindent%
FONSECA, R. L. \textbf{PsyChO: The Ball}. Trabalho de Conclusão de Curso
 - Instituto de Matemática e Estatística, Universidade de São Paulo,
São Paulo, 2017.
\\

Desenvolvimento de jogos é uma área da computação repleta de desafios. O jogo digital PsyChO: The Ball, um \textit{Top Down Shooter} psicodélico minimalista, foi produzido, a partir do inicio, utilizando apenas software livre. Feito no arcabouço \textit{LÖVE} com a linguagem Lua, o processo de desenvolvimento passou por todas as etapas necessárias para uma produção de jogos. Ele foi inspirado no Game Design dos jogos digitais \textit{Hotline Miami}, \textit{Touhou} e \textit{Hexagon} e, entre suas melhores características, temos sua jogabilidade frenética e efeitos especiais \textit{juicy}.
\\

\noindent%
\textbf{Palavras-chave:} desenvolvimento de jogos, game design, juiciness.
