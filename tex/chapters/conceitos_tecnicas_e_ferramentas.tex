% Referencia: MDA. Engeenering emergency (dorman's)

%conceitos:falar de gameplay, falar de gamedesign em geral, jogos em focado em experiencia, falar do jesse schell. Falar sobre game engine, game loop. parte grafica, audio, input. Dividido em cenas. falar do GDD
%tecnicas: falar de metodologias agil, versionamento de codigo, pratica de release. Documentação
%ferramentas:love, git, gimp, LMMS, github, google docs.

%sites uso footnotes. SObre artigos e outras coisas escritas eu uso cite


% ---------------------------------------------------------------------------- %
\chapter{Conceitos, Técnicas e Ferramentas}
\label{cap:conceitos_tecnicas_e_ferramentas}
% ---------------------------------------------------------------------------- %

Assim como no processo de desenvolver um software, um jogo possui vários conceitos e ferramentas próprias que ajudam na comunicação e organização durante o desenvolvimento.
Para familiarizar o leitor com as técnicas e jargões normalmente utilizados, vamos descrever nesse capítulo um pouco da linguagem específica no mundo de \textit{game development}, além das técnicas e ferramentas específicas que foram utilizadas na criação de \textit{PsyChO: The Ball}.

% ---------------------------------------------------------------------------- %
\section{Conceitos}
\label{sec:conceitos}
  A interatividade de um jogo consiste em que ações um usuário pode realizar, e de que forma isso altera o estado atual do jogo. Essa dinâmica define a jogabilidade de um jogo virtual, também chamada de \textit{gameplay}. Os tipos de interações que o jogador tem acesso são denomidas as \textit{mecânicas} do jogo, e estas variam dependendo de qual experiência você deseja transmitir para o usuário.

  Por exemplo em jogos de plataforma 2D clássicos como \textit{Mario} você controla um personagem que precisa pular e atravessar obstáculos até terminar o jogo. Para isso o jogador pode correr e pular, estas sendo as mecânicas fornecidas ao jogador para enfrentar os desafios propostos. Essas mecânicas juntas ao \textit{feedback} que o jogo retorna vão gerar uma \textit{experiência} para o jogador, e em geral toda a criação de jogos gira em torno de criar uma certa experiência desejada (como diversão, emoção, medo, aprendizado, etc).

  Existe uma área de estudo sobre a criação de experiências em jogos: \textit{Game Design}. Game Designer Jesse Schell define Game Design em seu livro sobre o assunto \cite{jessegamedesign} da seguinte forma:

  \begin{displayquote}
    \textit{Game Design é o ato de decidir como um jogo deve ser.}
  \end{displayquote}

   É natural esperar que essa área não seja uma ciência exata, de forma que não existe algoritmos ou fórmulas matemáticas que definem exatamente como fazer um jogo "bom". Muito pelo contrário, o estudo na área de Game Design é um conjunto de ideias e linhas de pensamento para auxiliar no desenvolvimento de jogos e mais facilmente atingir a experiência desejada que o desenvolvedor quer transmitir. Também no mesmo livro de Schell \cite{jessegamedesign} ele cria um sistema de "lentes", cada uma representanado alguma visão ou crítica para analisar as mecânicas e funcionalidades de um jogo e averiguar a consistência entre experiência desejada e transmitida.

  Como temos esse objetivo final de \textit{experiência para o usuário}, é muito comum a utilização de documentos para formalmente específicar e detalhar como será o desenvolvimento do jogo. Um documento famoso desse tipo é o \textit{Game Design Document} ou \textit{GDD}. Seu objetivo é servir como uma "receita de bolo", determinando o gênero do jogo, suas mecânicas, detalhamentos de personagens ou roteiro da história, ou até uma breve descrição do que é seu jogo, de forma que todos que trabalhem no projeto possam facilmente consultar e utilizar como guia durante o desenvolvimento. Em seu livro de Fundamentos em Game Design \cite{ernestgamedesign}, Ernest Adams descreve Game Design Documents da seguinte forma:

  \begin{displayquote}
    \textit{Os documentos gravam decisões feitas and concordadas oralmente [entre os integrantes do grupo]; (...) mais importante que isso, eles transformam ideias vagas em planos explícitos}
  \end{displayquote}

  Desta forma, apesar de parecer um gasto adicional de tempo sem grandes benefícios, a manutenção de documentos de design acaba poupando tempo em resolução de conflitos de ideias e deixa o desenvolvimento de um jogo muito mais fluído.\\

  Na parte prática de desenvolvimento vamos precisar de ferramentas para produzir e gerar o jogo. Estas são chamdas de \textit{Motores de Jogos} ou \textit{Game Engines}. Uma \textit{Game Engine} consiste em várias bibliotecas que auxiliam na produção de um jogo, além de alguma estruturação para gerenciar e manipular elementos lógicos. Elas possuem métodos para desenhar objetos na tela, reproduzir arquivos de sons, lidar com input de usuário e em muitos casos outras funcionalidades úteis como métodos para simular interações físicas ou gerenciar janelas. Uma versão mais simples de uma \textit{Game Engine} é uma \textit{Framework} ou \textit{Arcabouço}. Estas são um conjunto de bibliotecas mas geralmente mais abertas  e sem uma estrutura rígida para desenvolver o jogo.

  Mas seja uma Game Engine ou um Arcabouço, todas ferramentas para criação de jogos utilizam um laço lógico chamado de \textit{Game Loop}. Cada ferramenta tem sua própria versão e implementação, mas em geral todas tem 3 pontos chaves: processar inputs, atualizar objetos no jogo e renderizar elementos gráficos. Podemos visualizar esse laço da seguinte forma, como visto no segundo capítulo do livro de Robert Nystrom, Game Programming Patterns \cite{robertgameloop}:

  \begin{lstlisting}
    while (true)
    {
      processInput();
      update();
      render();
    }
  \end{lstlisting}

  A função textit{processInput} é responsável por verificar se o usuário apertou alguma tecla ou de algum jeito fez algo no mundo físico que afete o jogo. A função \textit{update} é responsável pela parte lógica do jogo, como atualizar as posições de inimigos, determinar colisões e qualquer outra atualização lógica necessária. É muito comum que essa função receba um argumento especial que diz quanto tempo se passou desde a últiam chamada dela (muitas vezes chamado de \textit{dt}). Assim é possível fazer cálculos relativos ao tempo de forma mais realística. Por último a função \textit{render} é responsável por desenhar os objetos na tela, normalmente apagando o que foi desenhado anteriormente.


% ---------------------------------------------------------------------------- %
\section{Técnicas}
\label{sec:tecnicas}

    Texto texto texto texto texto texto texto texto texto texto texto texto
    texto texto texto texto texto texto texto texto texto texto texto texto
    texto texto texto texto texto texto texto.


    Texto texto texto texto texto texto texto texto texto texto texto texto
    texto texto texto texto texto texto texto texto texto texto texto texto.

% ---------------------------------------------------------------------------- %
\section{Ferramentas}
\label{sec:ferramentas}

    Texto texto texto texto texto texto texto texto texto texto texto texto
    texto texto texto texto texto texto texto texto texto texto texto texto
    texto texto texto texto texto texto texto.


    Texto texto texto texto texto texto texto texto texto texto texto texto
    texto texto texto texto texto texto texto texto texto texto texto texto

% ---------------------------------------------------------------------------- %
