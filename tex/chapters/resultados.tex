% ---------------------------------------------------------------------------- %
\chapter{Resultados}
\label{cap:resultados}
% ---------------------------------------------------------------------------- %

Em dezembro de 2016 o protótipo de \textit{PsyChO: The Ball} foi finalizado, e teve seu primeiro lançamento em um evento expositivo da \textit{UspGameDev}. Nele vários alunos da \textit{Universidade de São Paulo} puderam comparecer e jogá-lo gratuitamente, dentre outros jogos de membros do grupo de extensão. Foi uma imensa satisfação observar outras pessoas jogando e se divertindo com o jogo, mesmo ele sendo um protótipo.

Uma das maiores importâncias nesses eventos é poder receber \textit{feedback} de outras pessoas. Estes variam dentre opiniões sobre mecânicas, descobrir se os conceitos de design escolhidos cumpriram seu propósito, ou até mesmo descobrir novos \textit{bugs} durante o evento. Isso facilita imensamente o processo após uma exposição de polir e consertar onde for necessário.

% ---------------------------------------------------------------------------- %
\section{Conteúdo}
\label{sec:conteudo}

\begin{figure}[h!]
\includegraphics[scale=.3]{ss1}
\centering
\caption{Primeiro chefão do jogo atacando o jogador}
\end{figure}

O jogo em seu estado atual possui 2 níveis dos 5 planejados. Cada nível tem sua própria trilha sonora, e é divido em quatro partes únicas, sendo a última composta por um \textit{chefão}. O jogo possui vários efeitos especiais visuais e sonoros, e abrange 5 inimigos diferentes para desafiar o jogador.

\begin{figure}[h!]
\includegraphics[scale=.3]{ss4}
\centering
\caption{Jogador morrendo para uma onda de inimigos lhe atacando}
\end{figure}

Além disso o jogo salva as maiores pontuações entre partidas, então jogadores podem sempre tentar superar a pontuação de colegas, ou tentar aumentar seu próprio recorde.
O sistema para rodar níveis através de scripts permite facilmente que usuários criem e joguem seus pŕoprios níveis sem precisar de muito conhecimento computacional.

% ---------------------------------------------------------------------------- %
\section{Eventos}
\label{sec:eventos}


\begin{figure}[h]
\includegraphics[scale=.06]{letsplay1_2}
\centering
\caption{Evento \textbf{Let's (test) Play!}. No fundo \textit{PsyChO: The Ball} - 02/12/2016}
\end{figure}

A primeira exposição pública de \textit{PsyChO: The Ball} foi em dezembro de 2016, no evento aberto organizado pelo grupo \textit{UspGameDev}: \textbf{Lets (test) Play!}. Nele todos integrantes do grupo tiveram uma chance de mostrar todo o progresso de desenvolvimento em seus projetos para a comunidade uspiana. Desta forma alunos, professores e funcionários puderam jogar, darem opiniões e acima de tudo se divertirem durante uma tarde nos jogos.

Foi um momento marcante no desenvolvimento de \textit{PsyChO: The Ball}, pois receber o feedback de outras pessoas traz tanto uma utilidade para o aperfeiçoamento do jogo, quanto uma satisfação pessoal ao ver pessoas se envolverem num projeto que demorou meses para ser desenvolvido. Uma das maiores mudanças que surgiu nesse primeiro evento foi a implementação do sistema para armazenar pontuações máximas, assim jogadores podiam comparar resultados no jogo entre sí.

O segundo grande evento ocorreu em julho de 2017, \textbf{II Let's (test) Play}. Nesta sequência do primeiro evento foi apresentado mais projetos de membros da \textit{UspGameDev}. Além disso teve uma quantidade maior de pessoas comparecendo, até mesmo de pessoas não atreladas à \textit{Universidade de São Paulo}. Foi possível mostrar todo o progresso no desenvolvimento do jogo nos 6 meses que se passaram, e novamente se fortalecendo com todo o feedback recebido pelos alunos e professores que quiseram jogar \textit{PsyChO: The Ball}.

\begin{figure}[h!]
  \includegraphics[scale=.06]{letsplay1_1}
  \centering
  \caption{Aluno jogando \textit{PsyChO: The Ball} no \textbf{Let's (test) Play!} - 02/12/2016}
\end{figure}

\begin{figure}[h]
\includegraphics[scale=.06]{letsplay2}
\centering
\caption{Foto do evento \textbf{II Let's (test) Play} - 05/07/2017}
\end{figure}

% ---------------------------------------------------------------------------- %
\section{Como Jogar}
\label{sec:how_to_play}


\textit{PsyChO: The Ball} possuí todos seus lançamentos disponiveis online em seu próprio repositório no domínio \textit{Github}:
\\~\\
\url{https://github.com/uspgamedev/Project-Telos/releases}
\\~\\
Neste link é possível encontrar instruções de como instalar e jogar \textit{PsyChO: The Ball} nos sistemas operacionais \textit{Linux} (suporte direto para distros baseadas em \textit{Ubuntu} ou \textit{Debian}), \textit{Windows} ou \textit{Macintosh}.
\\~\\
Além disso é possível acessar diretamente a versão mais atualizada do código fonte através da página principal do repositório do projeto:
\\~\\
\url{https://github.com/uspgamedev/Project-Telos/}
